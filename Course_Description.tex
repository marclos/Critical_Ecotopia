% Course Description Fall 2016

% \pagestyle{empty}

%\renewcommand{\thefootnote}{\fnsymbol{footnote}}

% \begin{picture}(6,.1) 
% \put(0,0) {\line(1,0){4.25}}         
% \end{picture}

\section{Course Description} 

We begin a journey to explore possible futures for this human-dominated world.\sidenote{What's your ecotopia?} From creation myths to visions of apocalypse, we cast our hopes and fears into stories that reflect the long and complex relationship between humans and the natural world. We will examine variety of ecological utopias and dystopias and consider how they use (or misuse) scientific knowledge and cultural fears using a wide range of sources, which include utopian narratives, visual arts, science fiction film, and concrete attempts to create in utopian communities. Our sources include fictional works (e.g. More's Utopia, Gilman's Herland), film (e.g. On the Beach, The Day After Tomorrow, Children of Men, and The Hunger Games), various intentional community descriptions (e.g. communes, kibbutz). By drawing on these sources, we will evaluate we might imagine the relationship between ``us," ``them,'' and ``the world,'' and how this triangle of actors continues to shape contemporary thought about our ecological context. 

\section{Rationale}
The Utopia genre has a long history in western culture, but with each new period and generation, these ideas have been conceptualized in new ways---in part to address a new appreciation of some sort of social ill.\sidenote{Why do Ecotopia's matter?} By understanding the relationships between utopia/dystopia ``geographies'' and the reality as we perceive it, we might be able to decipher if there are some common threads, hopes, and assumptions that might be used to inform how we think of our shared future as humans.

\section{Goals} 

ID1 emphasizes helping you become active participants in your own educations, encouraging you to think critically, to use writing as a way to aid in that thinking, and to learn from the perspectives and experiences of others.\sidenote{What do you want to get out of this class?} By the end of the semester, I hope you will have developed an idea that you own. Faculty do not lecture or give exams, and student discussions are the center of most class meetings.

%\vskip.25in
\paragraph{Course Learning Outcomes} 

This course teaches critical thinking and much of the evidence is based on your writing and participation in the course.\sidenote{Why are these skills useful?} The following course learning outcomes are the skills that we will focus on for the course: Pomona students should be able to engage the work and ideas of others; to articulate nuanced, reflective positions and present them in a sustained, persuasive manner to a specific imagined audience.

To this end, I have translated these outcomes and aligned them with an assessment rubric, so by the end of this course, you should be able to:

\begin{enumerate*}
  \item Write an effective academic essay that has the following characteristics:
    \begin{itemize*}
      \item A well-defined and intentionally stated thesis;
      \item logical structure that builds coherent arguments;
      \item sufficient, appropriate, and interesting evidence to support each argument;
      \item direct engagement of counter arguments;
      \item source material that is analyzed in an original and insightful manner;
      \item external sources which are used effectively;
      \item consistent use of a sophisticated and academic style; and
      \item follow the conventions in academic writing in terms of mechanics;
    \end{itemize*}
  \item Contribute to seminars with knowledgeable and accurate use of texts, where claims are justified with clear arguments and use of evidence; and
  \item Provide effective and constructive peer review.
\end{enumerate*}

%\vskip.25in

\section{Instructor Information}
\noindent \emph{Instructor:} Professor Marc Los Huertos

\noindent \emph{Office:} Seeley G. Mudd, Room 130A

\noindent \emph{Office Hours:} Monday 10:00-11:00 AM; Thursday 1:00-2:00 PM; or by arrangement.

\noindent \emph{Email:} marc.loshuertos@pomona.edu\sidenote{Please reserve the use of email to focus on logistic questions or concerns. In the last few years email has become an untenable method of communicating course content questions. Please use my office hours to address course content questions. In addition, do not turn in assignments via email.}

\noindent \emph{Phone:} 909-607-7787 (in person meetings are best.)

\section{Meeting Times and Location} The seminar will occur Tuesdays and Thursdays 11:00-12:15 PM in Lincoln 1135. 

%\section{Expectations}
%\subsection{What do I expect from you?}
%\paragraph{Effort} I have designed this project-based course to provide you an opportunity to develop into an environmental scientist. To improve your success, you will need to take the initiative in your learning. The course requirements are diverse; and your success is greatly enhanced if you are engaged each week. If you fall behind or struggle with the material, please take the opportunity to meet with me in office regularly and often. 

%\paragraph{Respectful}Be prepared; be consciously safe; clean up after yourself; and share the work load with partners.

%\paragraph{Prepared}Since course activities include laboratory and field work through the semester, you will need to come to class everyday prepared. This means that you will need to have done the reading (and in some cases completed a quiz), bring lab safety glasses, and wear long-pants and closed-toed shoes. Safety glasses can be obtained on-line or at hardware stores, such as Home Depot. You will need a bound laboratory and field notebook (no spiral notebooks). You will also need a calculator. Any scientific calculator is fine (e.g. Texas Instruments TI30 $\sim$\$25).

%\paragraph{Professional}Regular attendance;\footnote{You are expected to attend every class. If you do not turn in a homework assignment, take a quiz, or take an exam at the scheduled time, you will not be able to complete the assignment later. However, an absence due to unusual and documented circumstances can be excused when the instructor is notified \emph{as soon as possible}. Generally, being sick is not unusual.  Notification is almost always possible immediately upon occurrence of unusual circumstances.  Failure to make such timely notification may result in denial of a request to make up work. A college-sanctioned excuse from a scheduled class activity, i.e. an exam, must be presented in writing no later than one week prior to the date of the absence.}
%be on-time; 
%turn-in your assignments; 
%and use all available resources (textbook, handouts, course web-sites, library texts and resources, and peer-reviewed journal articles).

%\paragraph{Integrity} Turn in your own work. Not only is plagiarism and other forms of academic dishonesty not tolerated, plagiarism will also not give you the skills you need once you graduate. In addition, students who commit plagiarism on an assignment will fail the course--at minimum. In addition, the student's name will be reported to the college administration with the potential for further disciplinary action.  

%\vskip.25in
%\subsection{What can you expect from me?}
%\paragraph{Real World Applications} Beginning in 2000, I realized my work could be described as environmental monitoring and assessment. All my research interests (nitrogen biogeochemistry, water quality monitoring, aquatic bioassessments, and greenhouse gas emissions) fit nicely into this umbrella term. Nevertheless, it has taken a number of years ($\sim$5) to embrace this concept in my teaching. Starting in 2006, I started teaching a course that would become Environmental Monitoring and Assessment. I work hard each summer to improve the course to provide hands-on and relevant content. Now that I am at Pomona College, I will be revising the course to meet the needs of EA students.


%\paragraph{Prepared} You can expect that I will arrive in class each day on time and prepared to facilitate a climate for learning. I have created a project-based course to enrich your understanding of environmental science. I will provide handouts regularly to augment the assigned readings. At some point, I hope to collate these resources into a single document such as a course textbook. But that will probably not happen for several years.

%\paragraph{Available} I will be available to meet with you during office hours or at whatever other times we arrange. I will take your concerns and interests seriously. If you feel that I have assessed you unfairly, please come and speak with me about it. 

%\paragraph{Respectful} I am always interested in your insights, experiences and feedback for the course. I value your contribution to the class and beyond as colleagues in Environmental Analysis.

\section{Course Resources} 

\noindent Required Texts: 
\begin{itemize*}
	\item More, T. 1999. Utopia. Hackett Publishing Company. ISBN: 9780872203761.
	\item Piercy, M. 1976. Woman on the Edge of Time. Fawcett. ISBN: 9780449210826.
  \item Callenbach, E. 1975. Ecotopia. Bantam Books. ISBN: 9780553348477.
  \item Gilman, CP. 1998. Herland Dover. ISBN: 9780486404295.
  \item Le Guin, U.  Dispossessed. Harper Collins. ISBN: 9780061054884.
  \item Schaer, R. et al. 2000. Utopia, Search in Western Society. The New Your Public Library/Oxford University Press. 386p. (out of print, on 2 hour reserve at the library).
\end{itemize*}

\noindent Recommended Resources (Buying or Borrowing)



\section{Writing Resources}

In lieu of a required standard grammar and style handbook, students are encouraged to become familiar with the extensive online resources available through the Purdue University Online Writing Lab (OWL): \url{http://owl.english.purdue.edu/}. Links to the Purdue OWL have been installed on our course Sakai site.\sidenote{How are good ways of using writing resources?}

We will use a range of resources to develop our writing skills. These include an in-house writing assistant (Ki'Amber Thompson, XXX@pomona.edu) who will be working with you throughout the semester on your writing for this course.\sidenote{How might a writing fellow help?}

In addition to our own in-house writing assistant, the Pomona College Writing Center (on the ground floor of Smith Campus Center across from the Living Room) offers students free, one-on-one consultations at any stage of the writing process from generating a thesis and structuring an argument to fine-tuning a draft. They also work with students on all aspects of oral presentations. 

Pomona students majoring in subjects including Economics, Computer Science, English, and History will work with you on an assignment from any discipline. Consultations are available by appointment, which you can make online: \url{http://writing.pomona.edu}.

The Writing Center also offers drop-in hours Sundays through Thursdays from 8-10 p.m.

\section{Library Resources}

We also have a class website designed by the our class Librarian:\sidenote{What do libarians do?} Jessica Greene. She can be reached by email: \url{jessica_greene@cuc.claremont.edu} or by phone, 909-607-3892.

The website URL is \url{http://libguides.libraries.claremont.edu/ID1-Utopia}.

\section{Important Dates} 
Last day  to drop the course is \emph{Thursday, October 20}. Be sure to check with me before the drop date if you are concerned about passing the course. There is no final and no class during finals week. 

\section{Learning Diversity Accommodations} 

Pomona College welcomes and accommodates students with disabilities as part of campus diversity and to ensure legal compliance. Students with disabilities should notify me in person or by email if they need accommodations. ALSO, see \url{http://www.pomona.edu/administration/dean-of-students/disability-accommodations/learning-disabilities.aspx} for more information.\sidenote{Why are accommodations an important component of education?}

%\vskip.25in
\section{Course Improvement}
Project-based course require attention to the process and being willing to make adjustments in the project management. In contrasts to courses that have activities with pre-determined outcomes, this effort (workload) required will vary dramatically from week to week and between year to year. Please keep this in mind as the course develops and it will be up to your teams to develop a time management system. In addition, be sure to schedule times that you meet with me to discuss and reflect on the progress of the course so I can facilitate effective use of your time.\sidenote{Describe effective ways to communicate issues about the course?}

And even more generally, suggestions for improvement are welcome and is often key to the success of this type of course. Concerns about the course can be brought to my attention at any time. 

\section{Grading} 

This is a seminar style course. Thus, the course relies on engagement with the texts and active participation for each seminar meeting and writing. There are no written examinations in ID 1 and all written materials are due in class by the last day of classes, December 7. 

\subsection{Allocation of Points} 

The letter grade in the course will be based on classroom participation (20\%) and the four papers. Grading is weighted as follows:

\begin{table}
\begin{tabular}{llclr}\hline
Essay	& Brief Description	            & Max. Length	& Due Date	 & Weight \\\hline\hline
\#1	  & My Ecotopia	                  & 2	          & September 5	& N/NP \\
\#2	  & Parallel Visions              & 2           & September 8 & N/NP \\
\#3   & Entering a Conversation       & 5	          & September 20 & 	5\% \\
\#4   & The Modern Genre              & 5           & October 20  & 15\% \\
\#5   & Infotopia                     &	2	          & October 25	& 5\% \\
\#6	  & Anti-Utopia	                  & 5           &	November 17	 & 15\% \\
\#7	  & Utopia/Dystopia	              & 8           &	November 27	 & 20\% \\
\#8	  & Critical Ecotopia             & 8           &	December 7  & 15\% \\
    & Course facilitation	            &             &             &	5\% \\
    & Course participation	          &             &             & 20\% \\ \hline
\hline
\end{tabular}
\end{table}

Note: Paper \#7 will be submitted electronically.

\subsection{Participation}

Effective participation in the course requires that you come to class prepared, having completed the reading, and ready to contribute to and learn from your peers. If you do not arrive prepared, the instructor may ask you to leave to improve the seminar content and reduce distractions. For primary source readings, you will be complete an active reading checklist to promote reading comprehension, which will be included in your participation grade as ``completed'' or ``incomplete.''

The instructor reserves the right to lower the final grade because of poor class attendance and/or lack of preparedness. If you are unable to attend class, consult the instructor before the date of absence.

All papers will be submitted as hard-copies on the class due date. \emph{Late assignments will not be accepted.}\footnote{Why is this? Are there valid reasons?} 

\subsection{Letter Grades}

I hold students to high expectations in this course, but provide as much support as possible for you to succeed. After several years of teaching, I articulate what the grading structure means:\sidenote{Describe the types of effort that might be associated with each grading category?}
  
\begin{itemize}
	\item Grade A represents exceptionally high achievement as a result of effort and intellectual initiative. The course learning outcomes were consistently exceeded and the work submitted could be used as models for other students to follow.
	\item Grade B represents a high achievement as a result of ability and intellectual initiative. The course learning outcomes were consistently exceeded but the work submitted could not be used as models for other students to follow.
	\item Grade C represents the minimum required achievement to meet the learning outcomes on a consistent basis. 
	\item Grade D represents the minimum passing grade and the performance includes consistently unmet learning outcomes. 
	\item Grade F represents unsatisfactory performance as a consistent failure to meet the learning outcomes and indicates failure in the course.
\end{itemize}

Doing well in the course requires effort, and I try to make the path to success as transparent as possible. However, if you feel like you are struggling, please contact me as soon as possible, and we can see how to proceed. If you find that you are unable to complete assignments on-time, consider dropping the course as a last resort. Last day to drop the course is \textbf{October 20, 2016}. 

\paragraph{Extra Credit} No extra credit will be made available in this course. Your success in the class is based on the assigned assessments.\sidenote{Describe some reasons that might justify extra credit. Describe reasons why extra credit might be a problem in a classroom.}

Based on the total points in the course, the cutoffs percentages below are used to assign final grades. Please note that I will not use D+, D-, F+, F- as a final grade in this course. 

\begin{table}[htbp]
	\centering
\begin{tabular}{lrl} \hline
Letter Grade & Lower Range & Upper Range \\ \hline\hline
A$^+$  &$ \geq98\%$ & --		 \\
A      & $\geq93\%$ & $<98\%$\\
A$^-$  & $\geq90\%$ & $<93\%$\\
B$^+$  & $\geq87\%$ & $<90\%$\\
B      & $\geq83\%$ & $<87\%$\\
B$^-$  & $\geq80\%$ & $<83\%$\\
C$^+$  & $\geq77\%$ & $<80\%$\\
C      & $\geq73\%$ & $<77\%$\\
C$^-$  & $\geq70\%$ & $<73\%$\\
D      & $\geq60\%$ & $<70\%$\\
F      & $\geq0\% $ & $<60\%$\\ \hline
		\end{tabular}
\end{table}

Finally, intellectual integrity plays a central role in your education. In addition, science for environmental policy requires integrity or the science is discounted. You are paying for your education, and what you invest will translate to what you receive from your experience at Pomona College. Unfortunately, some people need more motivation. Cheating and plagiarism do occur in scholarly work. They both engender professional dishonesty. I will hold students to the highest level of professional integrity. Cheating or committing plagiarism will result in the fail this course and be reported to the administration for possible further disciplinary actions as outlined by campus policy.


\paragraph{What is plagiarism?} Plagiarism is the use of someone else's material and claiming as your own. In science, it is easy to avoid plagiarism by simply rephrasing and citing the author who has the authority to make the claims we need for our arguments. So, we will spend some time discussing how to properly cite authors. As a first rule of thumb, do not use any written or web-based resource as scientific evidence unless it is from a peer reviewed article. Most web-based content does not qualify as peer-reviewed content; however, these resources can be used to further your personal understanding or to help you locate scientific evidence. For this course, Web-site references are prohibited sources when turning material in.\sidenote{Describe some ways that you might  be able to avoid any perception of plagiarism?}

